Como conclusión queremos mencionar que la implementación de nuestro \texttt{traceroute} nos resultó sencilla, fruto de las facilidades que nos otorgó la librería \texttt{scapy} a la hora de armar, enviar y recibir paquetes utilizando el protocolo \texttt{ICMP}. Sin embargo, nos hubiera gustado llegar a implementar un \texttt{traceroute} utilizando los flags Option de los paquetes \texttt{IP}, cuyas ventajas sobre una implementación basada en \texttt{ICMP} es la menor cantidad de paquetes que necesita para funcionar y el hecho que establece una ruta única al destino. Ésta limitación se debió a la, prácticamente, inexistente documentación de la librería scapy, razón por la cual nos resultó muy dificultoso la implementación mediante prueba y error.

Por otro lado, nos parece importante mencionar, a partir de los resultados de nuestros experimentos, lo común que resultan las diversas anomalías en la red. En particular los \textit{false RTT} y los \textit{missing hops} fueron las anomalías con mayor frecuencia de aparición. Entendemos que esta problemática dificulta mapeos de la red y dificulta la detección y solución de diversos problemas, como los mencionados por Jobst en su artículo\cite{Jobst}.

Por último, nos resultó curioso las discrepancias entre la ubicación posible de los host inferida a partir de los valores de \texttt{RTT} del mismo y la ubicación de los mismos dadas por herramientas de geolocalización. Concluimos que estas diferencias se deben a como están implementadas las herramientas de geolocalización, las cuales parecería que relacionan ubicación de un host con los prefijos en su dirección IP, sabiendo que rango de direcciones pertenecen a que organización.
