En la siguiente sección vamos a mostrar y describir la gramática construida para el lenguaje \textit{Dibu}, llamaremos a esta
gramática G:\\
\\
\textbf{Producciones}:
\newline
P $\rightarrow$ S P $\mid$ $\lambda$\\
S $\rightarrow$ id PARAMS\\
PARAMS $\rightarrow$  id = V, PARAMS $\mid$ id = V\\
V $\rightarrow$ num $\mid$ string $\mid$ (num, num) $\mid$ [ARRAY]\\
ARRAY $\rightarrow$ (num, num), ARRAY $\mid$ (num, num)\\
\\
\textit{G} = \{\{P, S, PARAMS, V, ARRAY\}, \{num, string, (, , , ), [, ], =, id\}, Descripto por la gramática, P\}\\

Como se describe en el enunciado, el lenguaje \textit{Dibu} es una serie de instrucciones de la forma:\\
\\
IDENTIFICADOR PARAM1=V1, PARAM2=V2, ..., PARAMN=VN\\

Nuestra gramática consiste en cinco s\'imbolos no terminales, que generan 11 producciones:
\begin{itemize}
	\item Las producciones del no terminal ARRAY describen la secuencia de pares numericos que van dentro de un arreglo.
	\item Las producciónes del no terminal V (valor) contiene las combinaciones de terminales que se pueden esperar como valores de los parámetros (num, string, point, array).
	\item PARAMS posee la secuencia compuesta por: parámetro = valor.
	\item S describe una figura, con su nombre y parámetros.
	\item P describe la serie compuesta de las instrucciones descriptas en el no terminal S.
\end{itemize}

Con estas producciones se puede ver que nuestra gramática describe precisamente la serie de instrucciones de \textit{Dibu}. Se debe
aclarar que esta gramática no respeta todas las restricciones de \textit{Dibu}, como por ejemplo que solo aparezca la
instrucción size una vez. Estas cuestiones se tratarán en el lexer y el parser.\\

Por otro lado tenemos que definir que tipo de gramática es \textit{G}. Para lograr este objetivo no realizaremos una demostración rigurosa
aunque si analizaremos la misma descartando, en primera instancia, que la gramática sea ambigua. Posteriormente veremos a que tipos de gramáticas no pertenece \textit{G}.\\

Esta gramática no es LL(1) porque en las producciones de PARAMS, por ejemplo, sucede que la intersecci\'o de s\'imbolos distinguidos no es vac\'ia:

SD(PARAMS $\rightarrow$ id = V, PARAMS) $\cap$ SD(PARAMS $\rightarrow$ id = V) = {id}.\\

La gramática tampoco es LR(0). A continuación se mostrara el autómata correspondiente a la gramática, marcando los conflictos.
Solamente se consideraron los estados del autómata que fueran interesantes por sus conflictos.\\

\includegraphics[scale=0.5]{imagenes/tleng.png}


La gramática es SLR ya que conservando el mismo autómata vemos que los siguientes de P y ARRAY que son los implicados en
las nodos que generan los conflictos son:\\

Siguientes(P) = \{\$\}
Siguientes(ARRAY) = \{ ] \}\\

Con lo cual los nodos 1 y 2 que presentaban conflictos por el lambda quedan resueltos debido a que solamente se reducirá por
\$ y no por id. Por último el nodo conflictivo 3 solo reducirá con ] por lo que se eliminan todos los conflictos shift/reduce.
