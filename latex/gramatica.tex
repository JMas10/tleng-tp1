En la siguiente sección vamos a mostrar y describir la gramática construida para el lenguaje \textit{Dibu}, llamo a esta
gramática G:\\
\\
P $\rightarrow$ S newline P $\mid$ $\lambda$\\
S $\rightarrow$ id PARAMS\\
PARAMS $\rightarrow$  id = V PARAMS P $\mid$ id = V\\
V $\rightarrow$ num $\mid$ string $\mid$ (num, num) $\mid$ [ARRAY]\\
ARRAY $\rightarrow$ [num, num], ARRAY $\mid$ [num, num]\\
\\
\textit{G} = \{\{P, S, PARAMS, V, ARRAY\}, \{num, string, (, , , ), [, ], =, id, newline\}, Descripto por la gramática, P\}\\

Como se describe en el enunciado, el lenguaje \textit{Dibu} es una serie de instrucciones de la forma:\\
\\
IDENTIFICADOR PARAM1=V1, PARAM2=V2, ..., PARAMN=VN\\

Nuestra gramática consiste en cinco producciones. Comenzando con ARRAY la cual describe la secuencia de valores numericos
separados por una coma, la cual se utilizara en V para describir arreglos. La producción V que contiene las combinaciones
de terminales que se pueden esperar como valores de los parámetros (num, string, point, array).
PARAMS posee la serie de compuesta por: parámetro = valor. S describe una instrucción, con su nombre y parámetros. Y
finalmente P, describe la serie compuesta de las instrucciones descriptas en el no-terminal S. Con estas producciones se
puede ver con estos que nuestra gramática describe precisamente la serie de instrucciones de \textit{Dibu}. Se debe
aclarar que esta gramática no describe todas las restricciones de \textit{Dibu}, como por ejemplo que solo aparezca la
instrucción size una vez. Estas cuestriones se tratarán en el lexer y el parser.\\

Por otro lado tenemos el tipo de la gramática, para exponer esta información no realizaremos una demostración rigurosa
aunque si analizaremos la misma descartando en primera instancia que la gramática sea ambigua. Posteriormente reafirmaremos
en que tipos de gramáticas no se cuenta y el por qué.\\

Esta gramática no es LL(1) porque las producciones de PARAMS no cumplen con el
requerimiento de estas gramáticas(SD(PARAMS $\rightarrow$ id = V PARAMS P) $\cup$ SD(PARAMS $\rightarrow$ id = V)).\\

La gramática tampoco es LR(0). A continuación se mostrara el autómata correspondiente a la gramática, marcando los conflictos.\\


\includegraphics[scale=0.5]{imagenes/tleng.png}


No es SLR ya que los siguientes de PARAMS son \{newline, id\} por lo tanto no se resuelve el conflicto ya que hay un
shift/reduce provocado por el terminal id aunque si resuelve los demás conflictos.\\

Finalmente nuestra gramática es LR(1), observemos que el primer conflicto quedaría resuelto ya que solo haría reduce en
 \$, en el caso del segundo se resolvería de la misma forma. En el caso del tercero, en una producción anterior arrastraría
 el token newline y terminaría reduciendo por este, eliminando así el conflicto provocado por id. Por último tenemos el cuarto
similar al anterior en las producciones anteriores se los asociaría con el token ] el cual eliminaría el conflicto con la coma.
